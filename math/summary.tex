\documentclass{article}
\usepackage[a4paper,top=0.5cm,bottom=1.5cm,left=1.5cm,right=1.5cm,marginparwidth=1.75cm]{geometry}
\usepackage{amsmath,mathtools}
\usepackage{graphicx}
\usepackage{mathrsfs}
\usepackage{braket}

\begin{document}
\Large
Let $n$ be the dimension size of a symmetric rank $r$ tensor.
Define the following quantities $\Delta_r$, which represent the number of unique elements in the 
upper hypertriangle of a tensor of rank $r$:
\begin{flalign*}
\Delta_0 &= 1 \\
\Delta_1 &= n \\
\Delta_2 &= \frac{n(n+1)}{2} \\
\Delta_3 &= \frac{n(n+1)(n+2)}{6} \\
\cdots   & \\
\Delta_r &= \frac{n(n+1)(n+2)\cdots (n + (r-1))}{r!} \\
\end{flalign*}

We now can define the functions $Z_r$, which map an element's multidimensional index  ($i \leq j \leq k ...$)
to the corresponding one-dimensional index in the vector containing the flattened upper hypertriangle of the tensor.

%Functions $Z_r$ which map a multidimensional index  ($i \leq j \leq k ...$) corresponding to the address of an element in the upper triangle of a symmetric tensor of rank $r$ and dimension size $n$   
%to the index corresponding to the same element in the \textbf{flattened upper-triangle vector}, 
%which is the row-wise ravelled (flattened) upper triangle of the symmetric tensor.
%In other words, the functions map the $r$-dimensional upper triangular symmetric tensor elements to the 1-dimensional vector of unique symmetric tensor elements.
%
%\begin{flalign*}
%\Delta_0 &= 1 \\
%\Delta_1 &= n \\
%\Delta_2 &= \frac{n(n+1)}{2} \\
%\Delta_3 &= \frac{n(n+1)(n+2)}{6} \\
%\cdots   & \\
%\Delta_r &= \frac{n(n+1)(n+2)\cdots (n + (r-1)}{r!} \\
%\end{flalign*}

\begin{flalign*}
Z_0(n)& = 0 \\
Z_1(i,n)& = Z_0(n-i) + \sum_{a=1}^{i} \Delta_0  \\
Z_2(i,j,n)& = Z_1(j-i,n-i) + \sum_{a=1}^{i} \Delta_1 - \sum_{b=1}^{a-1} \Delta_0  \\
Z_3(i,j,k,n)& = Z_2(j-i,k-i,n-i) + \sum_{a=1}^{i} \Delta_2 - \sum_{b=1}^{a-1} \Delta_1 + \sum_{c=1}^{b-1} \Delta_0  \\
Z_4(i,j,k,l,n)& = Z_3(j-i,k-i,l-i,n-i) + \sum_{a=1}^{i} \Delta_3 - \sum_{b=1}^{a-1} \Delta_2 + \sum_{c=1}^{b-1} \Delta_1  - \sum_{d=1}^{c-1} \Delta_0  \\
\end{flalign*}

\end{document}
